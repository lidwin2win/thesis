% Chapter 1

\chapter{\uppercase{Introduction}} % Main chapter title
\label{intro} % For referencing
\section{\uppercase{Motivation:}}
Technical Debt is a shortcut taken in a project development which is taken in order to speed up the delivery or completion of the project. Technical Debt as such is not harmful for example not commenting important information in some parts of the code and releasing it in a hurry is also a form of technical debt but the very crucial thing is after it is delivered the development team must make sure that they go back to the released skeleton and make sure that they comment it as soon as they get time after releasing the first version.
\par The real problem kicks in when after releasing the project if the team still does not add the left out comments, then future development of this code is hampered. So technical debt causes a problem when we let the debt to accumulate very similar to Financial Debt, in finance we make sure that the Debt we borrowed does not reach a humongous proportions, if it reaches such levels then we will not be able to pay off our Debts. Similarly if we do not periodically monitor and pay back our Technical Debts then it causes a huge disaster for companies.
\par But Technical Debt is not always identifiable, because sometimes by accident a Technical Debt is formed, so it is very hard to identify a Technical Debt in a software project.So it is becoming very important and pertinent for many companies around the world in the present to identify a very efficient and accurate method to identify a Technical Debt in a software project. Finding a accurate method which detects Technical Debt is the motivation of our project.
\section{\uppercase{Project Background:}}
For our project we are working on the Chromium Dataset that we got from Penn State University. Chromium is the open source counterpart of Google Chrome and the dataset are the bug issues of the browser Chromium, and there is a need for Chromium project to find out the bugs that have Technical Debt inherent with them, this is necessary because a bug due to Technical Debt causes a big software disaster and hence those bugs must be identified as soon as possible. At present Chromium project makes use of Software engineering experts to identify the Technical Debt bugs.
\par Chromium project has a open source issue tracker from where this data is extracted. Our Dataset consists of 700 issues out of which 200 issues are classified by software engineering experts and remaining 500 issues are classified by CMU. All 700 issues can be seen in the Issue-Tracker. We also have scraped out some issues from the issue tracker for our project. The 700 issue dataset has a binary classification label- either label T which indicates the description is Technical or NT which indicates that the issue is Non-Technical.
\par The dataset that we got has the following fields: Id, Author, Comments, Comments-Date, Comments-Label, Date, Description, Role, Status, Title, Type, Closed, Priority, Rating, Date, Label and Keywords. So we are applying our NLP and ML logic on this dataset.
\section{\uppercase{Necessity:}}
\begin{itemize}
\item At present the issues are being classified by Software Engineering experts. It takes a lot of time and money as it involves human Labour. 
\item Labelling each issue involves looking into a great amount of data about that issue which can be overwhelming.
\item So it is necessary that inorder to fasten this process there is need to automate this labelling process, for this we need to come up with a algorithm that reads a issue and then gives it a label which is as accurate as given by a human expert.
\end{itemize}
\section{\uppercase{Challenges:}}
\begin{itemize}
\item No one has tried this method of automating this process
\item We have to find a suitable feature selection method that will help in classification of these documents.
\end{itemize}
