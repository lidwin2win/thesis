% Chapter 2

\chapter{\uppercase{Literature Survey/Related Work}} % Main chapter title
\label{ch:survey} % For referencing

In order to understand about technical debt and the current methods of identifying this in a project there was an extensive Literature Review done, as part of the Literature Review many IEEE and ACM papers were read so that there is a complete understanding about the existing methods and the places where the method can be improved.
\section{\uppercase{The WyCash portfolio management system.}}
In the development of a software named WyCash+ for Wyatt Technology, there was a problem of a new feature fitting poorly in existing architecture, by using a makeshift method the new feature was accommodated and released. Later this makeshift method was replaced with the fully functional feature. Cunningham had first used the analogy of “technical debt” in this research paper.
\section{\uppercase{Tracking Technical Debt.}}
In this paper the authors had monitored a real MNC application. The application they monitored made use of MS-Exchange Server. At the time when they started work on this application the latest version of MS Exchange Server they had was 2003 version and there was the news that MS Exchange Server 2007 was going to release sooner or later, but they started developing the project with dependency on MS Exchange 2003. But unfortunately within 6 months of release the project MS Exchange 2007 version got released and hence they had to scrape out the released software thus causing a huge loss for the MNC. 
\section{\uppercase{Technical debt and agile software development practices and processes.}}
Agile software development process and practices have an effect on technical debt. Agile practices safeguarding software implementation have the most positive effect. Technical debt knowledge is implicit and hence the concept is under utilised.
\section{\uppercase{Identifying Self-Admitted Technical Debt in Open Source Projects Using Text Mining.}}
In this paper the authors made use of text mining to identify self admitted technical debt. Self admitted technical debt is the technique of adding information about technical debts in source codes by adding comments etc., Like for example someone may comment saying that “they have added a temporary variable as a shortcut” etc., Thus on mining these comments we must identify whether it happens to be a Technical Debt or not. They have made use of natural processing techniques in order to identify the Technical Debts in the source file.
\section{\uppercase{Detecting Technical Debt from Issue Trackers.}}
In this paper the authors made use of text mining on the issues that are found in the Issue trackers of the software. For this paper the authors had used the issues of a particular software whose issue can be got from the issue tracker. And then using a key word approach they extracted features and then gave it to Naive Bayes Classifier and thus got the required results.
